\chapter{Introduction}

Researchers and research institutions are faced with an increasing
demand for demonstrating their productivity, for example when applying
for funding. Demonstrating productivity means preparing publication
records, for inclusion in a CV, on a web page, or in an annual
report. Preparing such publication records is time-consuming and
error-prone, in particular when a large number of publications must be
presented in a uniform manner.

The bibliographic reference system \package{} solves this. Publication
records (in for example BibTeX format) from a large number of
researchers or departments may be imported into a common database,
validated against a list of known venues (journal and conference
names), checked for duplicate entries and common typos.  Publication
records may then be categorized and generated, in BibTeX, \LaTeX, or
PDF format.

The database is maintained as a simple text file which may be easily
edited using any text editor.

This manual describes the command-line interface of \package{}. In
addition, a programmer's interface is provided in the form of a Python
module. For documentation of the Python interface, refer to the Python
help system\footnote{\emp{help(publish)}}.
